\chapter*{Abstract}
In the last decade the growing interests about autonomous driving brings many researchers to spend  huge efforts to improve the interaction of the vehicles with the surrounding environment.
One of the key aspect in autonomous driving is the awareness of the surrounding provided by the map of the environment.
Indeed, a map is needed to plan a path and reach a specific destination, or to compare the current perception of the environment against a reference in order to estimate the position.
A suitable mapping, or reconstruction, algorithm needs  to be scalable, incremental and to estimate dense map. The scalability is needed for large-scale urban environments; an incremental algorithm allows to create or update a map  as new data are acquired; a dense map, or more precisely a continuous surface, enable a consistent and coherent navigability.
Computer vision and robotics communities create such kind of maps from images. 
A wide amount of algorithms have been proposed, but they focus on some aspects disregarding the others.
Many researchers, especially in computer vision, focus their reconstruction algorithms on accurate and dense results, only few works shows large-scale capabilities. These algorithms process all the images at the same time, disregarding any incremental processing. 
On the other side, in robotics the focus is mainly on incremental algorithms but the output maps are usually point clouds.
A very limited amount of work deal with incremental reconstruction of dense and continuous surfaces, but they are limited to small scale scenes.
In this thesis we propose a novel incremental reconstruction  pipeline that scalable, and it estimates manifold continuous dense meshes. 
We improve the accuracy of the state-of-the-art incremental reconstruction algorithm, that serves as initialization to a further photometric refinement step that we applied for the first time in incremental fashion. 
We also extended our work to jointly deal with laser range finders and images.
We tested our proposals against publicly available KITTI, Middelbury and EPFL datasets.

