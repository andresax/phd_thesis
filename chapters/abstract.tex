\chapter*{Abstract}
\vspace{-5pt}

 

In the last decade, the growing interest about autonomous driving brought many computer vision and robotics researchers to focus on the  vehicles understanding of the surrounding environment through a map of it.
A map is needed to plan a path to reach a specific destination, or to estimate the current position by comparing the current perception of the environment against a reference.
In robotics, a suitable mapping, or reconstruction, algorithm needs  to be scalable, incremental and to provide a dense map. 
Scalability is needed especially in large-scale environments; an incremental algorithm allows map update as new data are acquired; density enables a consistent and coherent navigability.
Researchers, in computer vision, focused their reconstruction algorithms on accurate and dense results, disregarding any incremental processing, and only few works shows large-scale capabilities.  
Instead, in robotics the focus is mainly on incremental algorithms but the output maps are usually point clouds; only a very limited amount of works estimate dense and continuous surfaces, but they are limited to small scale scenes.
As the main contribution of this thesis, we propose a novel incremental, automatic and scalable reconstruction  pipeline to estimates continuous dense manifold meshes; we especially focused on keeping the manifold property valid, to enable a coherent mesh refinement based on image appearance.  
Our contribution first improves the accuracy of the state-of-the-art incremental reconstruction algorithm both in case of video sequences of urban landscape, and in case of unordered set of images.
Then, to embed and refine automatically and incrementally new part of the scene in a reference  model, we proposed a novel mesh merging algorithm that preserves the manifold property.
Finally, we extended our work to jointly deal with laser range finders and images, exploiting the accuracy of the laser range measurement and the appearance provided by the images.
We tested our proposals against publicly available KITTI, Middlebury and EPFL datasets, which provide different scenarios in order to stress the flexibility of our approach.

