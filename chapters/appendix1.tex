\chapter{}
\section{Barycentric coordinates computation}
\label{app:barycentric_}
Here we describe how to compute the barycentric coordinates of a point $\mathbf{p}$ with respect to a triangle $t$.





% 
% float orientPoint(vec2 v0, vec2 v1, vec2 p){
%   mat2 m;
%   m[0][0] = (v1.x - v0.x);  m[0][1] = ( p.x - v0.x);
%   m[1][0] = (v1.y - v0.y);  m[1][1] = ( p.y - v0.y);
% 
%   return m[0][0] * m[1][1] - m[0][1] * m[1][0];
% }
% 
% vec3 barycentricCoordMine(vec2 p, vec2 p0, vec2 p1, vec2 p2){
%   vec2 v0, v1, v2;
%   //First Check if triangle is counter-clockwise
%   if(orientPoint(p0, p1, p2) > 0){
%     v0 = p0;
%     v1 = p1;
%     v2 = p2;
%   }else{
%     v0 = p0;
%     v1 = p2;
%     v2 = p1;
%   }
%   // Compute barycentric coordinates w.r.t pt1
%   vec3 barycentricCoordinates;
% 
%   float areaTrtwice = orientPoint(v0, v1, v2);
% 
%   if(areaTrtwice!=0){
%     barycentricCoordinates.x = orientPoint(v0, v1, p)/areaTrtwice;
%     barycentricCoordinates.y = orientPoint(v1, v2, p)/areaTrtwice;
%     barycentricCoordinates.z = orientPoint(v2, v0, p)/areaTrtwice;
%   }else {
%    return vec3(-1.0);
%   }