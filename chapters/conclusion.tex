\chapter[Conclusion and Future Works]{Conclusion and Future \\Works}
\label{ch:future_works}
In this thesis we presented a novel 3D reconstruction pipeline, capable of incrementally build a dense map of the perceived environment.
We focused our attention especially to the urban reconstruction, since autonomous driving is the primary application of the proposed techniques.
Therefore our a dense reconstruction algorithm needs scalability to reconstruct large-scale environments.


After a review of existing approaches, we chose to design a mesh-based reconstruction over the more common volumetric technique, both for its scalability and its accuracy. 
Mesh-based algorithms however are not fully automatic: the optimization algorithms adopted to evolve the surface according to the images, needs and initial manifold mesh. 
The classical initialization meshes suffer for non manifoldness which causes degeneracies in the evolution process. 
To face this issue literature fix manually the sources of non manifoldness. 
Moreover,  state-of-the-art  needs the whole set of images, therefore incremental reconstruction is not feasible.

Exploiting  manifold mesh reconstruction algorithm from sparse data proposed in the last years, we proposed a  novel ray tracing that leads to more accurate 3D models. 
Sparse data usually adopted comes from Structure from Motion algorithm.  
Since we deal with man-made environment with sharp edges, and the mesh rely on a Delaunay triangulation; we investigated if features different from classical SIFT or FAST may lead to better results. 
We found that the estimation of the 3D position of the 2D Edge-Points, i.e., points belonging to image edges, represents a more convenient choice: 3D positions of Edge-Points induce the edges of the Delaunay triangulation to lay on real-world edges.

From this manifold mesh we were able to build an incremental pipeline, where new images are processed iteratively.
When new images  capture a new part of the scene, we estimate a local rough mesh and we extend the existing dense accurate mesh thanks to a novel manifold-preserving mesh merging algorithm; it they capture a region of the scene estimated previously, we further refine it.
The refinement step evolves the mesh in order to minimize the Normalized Cross Correlation (NCC) between images.

In this thesis we also investigated how to improve the manifold initialization in order to improve the accuracy and the speed of the refinement. 
We proposed to sweep the existing mesh in its neighborhood to look for 3D points that induce very high NCC among the images. Then, we  update the mesh to embed these new 3D points.
Finally we proved that our approach can be extended to work with lidar data with a further effort to deal with moving points and with transparent surfaces.



