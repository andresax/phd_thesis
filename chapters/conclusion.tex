\chapter[Conclusion and Future Works]{Conclusion and Future \\Works}
\label{ch:future_works}
In this thesis we presented a novel 3D reconstruction pipeline, capable of incrementally build a dense map of the perceived environment.
We focused our attention especially to the urban reconstruction, since autonomous driving is the primary application of the proposed techniques.
Therefore our a dense reconstruction algorithm needs to be scalable  to reconstruct large-scale environments.


After a review of existing approaches, we chose to design a mesh-based reconstruction algorithm instead of the more common volumetric ones, both for its scalability and its accuracy. 
Existing mesh-based reconstruction algorithms however are not fully automatic since the optimization procedure, adopted to evolve the surface according to the photoconsistency, needs and initial manifold mesh. 
Classical initialization meshes extracted automatically from the data suffer for non manifoldness which causes degeneracies in the evolution process. 
To face this issue literature approaches fix manually the sources of non manifoldness, and this is not acceptable in our case. 
Moreover, state-of-the-art techniques require the whole set of images, therefore, incremental reconstruction is not feasible.

Exploiting  manifold mesh reconstruction algorithms from sparse data proposed in the last years, we proposed a  novel ray tracing approach that leads to more accurate 3D models. 
Sparse data usually come from Structure from Motion algorithm; since we deal with man-made environment with sharp edges, and the mesh relies on a Delaunay triangulation, we investigated if features different from classical SIFT or FAST, usually exploited in SfM or SLAM, may lead to better results. 
We found that the estimation of the 3D position of the 2D Edge-Points, \ie, points belonging to image edges, represents a more convenient choice; indeed, 3D positions of Edge-Points induce the edges of the Delaunay triangulation to lay on real-world edges and this, in turn, results in a more accurate reconstruction.

From this manifold mesh we were able to build an incremental pipeline, where new images are processed iteratively.
When new images  capture a new part of the scene, we estimate a local, possibly rough, mesh and we extend the dense accurate mesh thanks to a novel manifold-preserving mesh merging algorithm; the images capture a region of the scene estimated previously, we further refine it.
The refinement step evolves the mesh in order to minimize the Normalized Cross Correlation (NCC) between images, and it has been designed to exploit GPU computing though OpenGL shaders.

In this thesis we also investigated how to enrich the manifold initialization in order to improve the accuracy and the speed of the refinement. 
We proposed to sweep the existing mesh in its neighborhood to look for 3D points that induce very high NCC among the images. Then, we  update the mesh to embed these new 3D points.
Finally we proved that our approach can be extended to work with lidar data with a further effort to deal with moving points and with transparent surfaces.

Some aspects of this thesis are worth to be investigated and extended; in addition to fine and automatic parameter tuning, we discuss here some future research directions.
\begin{itemize}
 \item Mesh resolution adapts to the density of the initial sparse point cloud, but after photometric refinement is almost uniform along the whole mesh. In some cases, for instance, if the scene contains flat surfaces, a multi-resolution mesh could be more appropriate, this would lead more compact maps and faster refinement.
 \item The reconstruction pipeline could take advantage from the accurate semantic segmentation algorithm proposed by the machine learning community; scene-depending priors could be embedded into the reconstruction process to obtain a more robust algorithm, in analogy to the work proposed in  \cite{savinov2016semantic,HaZa16}.
 \item In our proposal we adopted a space carving and region growing approach to extract a manifold mesh. Graph cuts represents an alternative method to extract mesh out of the Delaunay Triangulation and it has been widely investigated in literature. It would be interesting to investigate if the framework of graph cuts is able to estimate incrementally a manifold mesh and to compare it with the approach proposed in this thesis.
\end{itemize}




















