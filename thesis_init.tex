
% Set paper size
\usepackage[twoside=true]{geometry}


\geometry{
  paperwidth=17cm,
  paperheight=24cm,
  margin=2cm,
  top=2.3cm,
  bindingoffset=0.4cm
}
% For printing in a4
%\geometry{a4paper,
%  margin=3cm,
%  top=3.8cm,
%  bindingoffset=0.4cm
%}

%Uncomment this for final prints: this just enables printing on a4 paper
\usepackage[cam,center,a4,pdflatex,axes]{crop}

\usepackage{phdthesis}

\usepackage{fancyhdr}
\usepackage{color}
\usepackage{array}
\usepackage{mdwmath}
\usepackage{mdwtab}
\usepackage{amsmath,amssymb}
\usepackage{amsfonts}
\usepackage{cite}
\usepackage{graphicx}
% \usepackage{glossaries}
\usepackage{listings}
\usepackage{subfig}
\usepackage{booktabs}
\usepackage{latexsym}
\usepackage{color}
\usepackage{url}
\usepackage{bnf}
\usepackage{minitoc}
\usepackage{rotating}
\usepackage{multirow}
\usepackage{phdtitle}
\usepackage{paralist}
\usepackage{bibentry}
%\usepackage[algochapter]{algorithm2e}
\usepackage{lscape}
\usepackage{algorithmic}
\usepackage{algorithm}
\usepackage{longtable}
\usepackage[T1]{fontenc}
\usepackage{amsthm}
\usepackage{todonotes}
\usepackage[utf8]{inputenc}
\hyphenation{}

\lstset{tabsize=2,basicstyle=\footnotesize,breaklines=true}

\usepackage[english,italian]{babel}

\usepackage{ulem}


%\usepackage[bookmarks=true,pdftex=false,bookmarksopen=true,pdfborder={0 0 0}]{hyperref}
\usepackage[bookmarks=true,bookmarksopen=true]{hyperref}
\normalem

\hypersetup{pdftitle={Incremental, Photoconsistent and Continuous 3D Reconstruction}, pdfauthor={Andrea Romanoni},
  colorlinks=true,% hyperlinks will be coloured
  linkcolor=red,% hyperlink text will be green
  linkbordercolor=red,% hyperlink border will be red
  pdfborder={0 0 0}
}
\nobibliography*

\department {Department DEIB}
\phdprogram{Doctoral Programme In Computer Engineering}

\author{Andrea Romanoni}
\title{Incremental, Photoconsistent and Continuous 3D Reconstruction}
\advisor{Matteo Matteucci}
\tutor{Andrea Bonarini}
 \supervisor{Andrea Bonarini}
\titleimage{img/01_Polimi_centrato_BN_positivo_senzaSfondo}
\phdcycle{Year 2015 - XXVIII Cycle}

\DeclareMathOperator*{\argmin}{arg\,min}

% \def\etal{\emph{et al}\onedot}

\newcommand{\Rn}{$\mathbb{R}^n$}
\newcommand{\Rthree}{$\mathbb{R}^3$}

\newtheorem{mydef}{Definition}
\newtheorem{thm}{Theorem}

\providecommand\cs[1]{\texttt{\string#1}}

\newcommand{\etal}{\mbox{\emph{et al.\ }}}
\newcommand{\ie}{\mbox{,\emph{i.e.},\ }}
\newcommand{\eg}{\mbox{,\emph{e.g.},\ }}
%\pdfsuppresswarningpagegroup=1